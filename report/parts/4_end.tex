\chapter{Discussion}

\section{Evaluation}
Evaluation against objectives

\paragraph{Error and weight space}

\paragraph{Stripe problem}

\paragraph{Generalisation}
Evaluate generalisation potential proved in generalisation section

\section{Critical appraisal}
\paragraph{Gradient descent}

\paragraph{Greedy probing}
Also talk about sampling techniques

\paragraph{Simulated annealing}
\label{sec:eval_sim_annealing}
More complex implementations of SA may combine the so-called downhill simplex algorithm \cite{nelder1965} with SA such as in \textcite[p. 444-455]{press1992}, thereby introducing three additional hyperparameters.
In fact, \citeauthor{press1992} remark that ``there can be quite a lot of problem-dependent subtlety'' in choosing the hyperparameters, and that ``success or failure is quite often determined by the choice of annealing schedule'' \cite*[p. 452]{press1992}.

Generic framework, so could not implement custom annealing schedules with restarts, etc.
Furthermore, at what point is the algorithm `adjusted too much' to the problem?

\paragraph{The framework}
Compare with scipy.optimize and nevergrad; also explain that they do not specifically target neural networks

\section{Conclusions}

\section{Future work}