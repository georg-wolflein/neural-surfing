\chapter{Evaluation and critical appraisal}

\section{Error and weight space}
\todo

\section{Stripe problem}
\todo

\section{Generalisation}
\todo: evaluate generalisation potential proved in generalisation section

\section{Optimisation techniques}
\subsection{Gradient descent}
\todo
\subsection{Greedy probing}
\todo: also talk about sampling techniques
\subsection{Simulated annealing}
\label{sec:eval_sim_annealing}
More complex implementations of SA may combine the so-called downhill simplex algorithm \cite{nelder1965} with SA such as in \textcite[p. 444-455]{press1992}, thereby introducing three additional hyperparameters.
In fact, \citeauthor{press1992} remark that ``there can be quite a lot of problem-dependent subtlety'' in choosing the hyperparameters, and that ``success or failure is quite often determined by the choice of annealing schedule'' \cite*[p. 452]{press1992}.

\todo: generic framework, so could not implement custom annealing schedules with restarts, etc.
Furthermore, at what point is the algorithm `adjusted too much' to the problem?

\section{The framework}
\todo: compare with scipy.optimize and nevergrad; also explain that they do not specifically target neural networks

\chapter{Conclusions and future work}
\todo