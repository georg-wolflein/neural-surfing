\chapter{The framework}
\label{chap:framework}
\section{Design}

\begin{figure}
    \centering
    \begin{tikzpicture}[
        block/.style={
            draw, 
            rectangle, 
            minimum height=1.5cm, 
            minimum width=3cm, align=center
        }, 
        line/.style={->,>=latex'},
        label/.style={midway,fill=white}
    ]
        \node[block] (problem) {Problem};
        \node[block, right = 4cm of problem] (metric) {Metric};
        \node[block, below = 2cm of problem] (agent) {Agent};
        \node[block, below = 2cm of metric] (visualisation) {Visualisation};
        \node[block] at ($(agent)!0.5!(visualisation)+(0,-3cm)$) (experiment) {Experiment};
        
        
        \draw[line] (problem.east) -- (metric.west) node[above left] {0..*} node[label] {defines};
        \draw[line] (agent.north) -- (problem.south) node[below right] {1} node[label] {trains on};
        \draw[line] (experiment.west) -| (agent.south) node[below left] {0..*} node[label] {coordinates};
        \draw[line] (visualisation.north) -- (metric.south) node[below right] {0..*} node[label] {plots};
        \draw[line] (experiment.east) -| (visualisation.south) node[below right] {0..*} node[label] {shows};   
    \end{tikzpicture}
    \caption{}
\end{figure}

\begin{figure}
    \centering
    \centerline{
        \begin{tikzpicture}
            \begin{abstractclass}[text width=9cm]{Agent}{-3.5,0}
                \attribute{problem: Problem}
                \operation[0]{compile()}
                \operation[0]{fit(X: Tensor, y: Tensor, epochs: int, callbacks: list)}
                \operation{train(epochs: int, metrics: list): dict}
            \end{abstractclass}
            \begin{abstractclass}[text width=5cm]{GradientBasedAgent}{-5.5,-5}
                \inherit{Agent}
                \operation{fit(\dots)}
            \end{abstractclass}
            \begin{abstractclass}[text width=8cm]{GradientFreeAgent}{3.5,-5}
                \inherit{Agent}
                % \attribute{sampler : SamplingTechnique}
                \operation{fit(\dots)}
                \operation{compile()}
                \operation{predict\_for\_weights(weights: Tensor, X: Tensor): Tensor}
                \operation{predict\_for\_multiple\_weights(weights: Tensor, X: Tensor): Tensor}
                \operation[0]{choose\_best\_weight\_update(weight\_samples: Tensor, weight\_history: Tensor, output\_history: Tensor, X: Tensor, y: Tensor): Tensor}
            \end{abstractclass}
            \begin{class}[text width=2cm]{MSE}{-7,-11}
                \inherit{GradientBasedAgent}
                \operation{compile()}
            \end{class}
            \begin{class}[text width=2cm]{LWGLD}{-4.5,-11}
                \inherit{GradientBasedAgent}
                \operation{compile()}
            \end{class}
            \begin{class}[text width=5cm]{GreedyProbing}{-.25,-11}
                \inherit{GradientFreeAgent}
                \operation{choose\_best\_weight\_update(\dots)}
            \end{class}
            \begin{class}[text width=5cm]{SimulatedAnnealing}{5.25,-11}
                \inherit{GradientFreeAgent}
                \operation{choose\_best\_weight\_update(\dots)}
            \end{class}
            \begin{abstractclass}[text width=5cm]{SamplingTechnique}{5,0}
                \operation[0]{initialize(num\_weights: int)}
                \operation[0]{\_\_call\_\_(weights: Tensor): \dots}
            \end{abstractclass}
            \unidirectionalAssociation{SamplingTechnique}{sampler}{}{GradientFreeAgent}
        \end{tikzpicture}
    }
    \caption{UML class diagram of the \texttt{agents} module, showing only the (non-underscored) public methods. Note that ``LossWithGoalLineDeviation'' is abbreviated as LWGLD and the subclasses of \texttt{SamplingTechnique} have been omitted.}
\end{figure}


\section{Implementation}
\todo

\texttt{pdoc3} documentation