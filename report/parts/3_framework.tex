\chapter{The neural framework}
\label{chap:framework}

\section{Design}
The requirements specification in \ref{sec:requirements} makes it clear that the framework must be made up of the following components:
\begin{enumerate}
    \item an interface to specify a neural \textit{problem} (as a labelled dataset with an initial weight configuration);
    \item a mechanism to calculate problem-specific \textit{metrics} during training;
    \item an interface for implementing custom \textit{visualisations} that plot metrics in real time;
    \item an interface for implementing an \textit{agent} that can be used to train on a problem; and
    \item the notion of an \textit{experiment} which coordinates the training and reporting (visualisation) aspects.
\end{enumerate}
\begin{figure}
    \centering
    \begin{tikzpicture}[
        block/.style={
            draw, 
            rectangle, 
            minimum height=1.5cm, 
            minimum width=3cm, align=center
        }, 
        line/.style={->,>=latex'},
        label/.style={midway,fill=white}
    ]
        \node[block] (problem) {Problem};
        \node[block, right = 4cm of problem] (metric) {Metric};
        \node[block, below = 2cm of problem] (agent) {Agent};
        \node[block, below = 2cm of metric] (visualisation) {Visualisation};
        \node[block] at ($(agent)!0.5!(visualisation)+(0,-3cm)$) (experiment) {Experiment};
        
        
        \draw[line] (problem.east) -- (metric.west) node[above left] {0..*} node[label] {defines};
        \draw[line] (agent.north) -- (problem.south) node[below right] {1} node[label] {trains on};
        \draw[line] (experiment.west) -| (agent.south) node[below left] {0..*} node[label] {coordinates};
        \draw[line] (visualisation.north) -- (metric.south) node[below right] {0..*} node[label] {plots};
        \draw[line] (experiment.east) -| (visualisation.south) node[below right] {0..*} node[label] {shows};   
    \end{tikzpicture}
    \caption{High-level view of the relationships between components of the \texttt{nf} framework. The numbers at the arrow tips denote the number of entities involved on that side of the relationship.}
    \label{fig:nf_components}
\end{figure}
\ref{fig:nf_components} shows a high-level view of these components and their associations. 
The user should be able to provide their own implementations of problems, visualisations, agents, and metrics. 
All but the latter are provided in the framework as abstract Python classes, such that the user can implement their own functionality.

The design decision was made to provide the user with a different means of specifying metrics that is more user-friendly than having to inherit from an abstract base class.
Due to the fact that metrics are defined on a per-problem basis and a metric is a function that can be evaluated on a problem in order to produce some data (tensor), it was decided to allow the user to specify these problems as \textit{decorated functions} in their implementation of the abstract \texttt{Problem} class, as shown in \ref{lst:metrics}.
\begin{listing}[h]
\begin{minted}{python}
class MyNeuralProblem(Problem):
    # ...
    
    @Problem.metric
    def weights(self):
        # Code to obtain the current weight state...
        return weight_state
\end{minted}
\caption{Example of how a metric can be defined for a problem using a decorated function. Here, the metric will be called \texttt{weights}.}
\label{lst:metrics}
\end{listing}

Finally, the \texttt{Experiment} class is not abstract because that is the class responsible for coordinating how all the user-defined components work together.
The user would simply instantiate this class with a set of agents (each defined on a problem), and then call the \texttt{run\_server()} method with a list of visualisations to display. \ref{lst:experiment} provides a minimal example of this.
\begin{listing}[h]
\begin{minted}{python}
agents = {
    "Agent A": AgentA(MyNeuralProblem()),
    "Agent B": AgentB(MyNeuralProblem())
}

experiment = Experiment(agents)
experiment.run_server([
    Scatter2D("weights"),
    Scatter2D("output:1", "weights:0"),
    Histogram("output:0")
])
\end{minted}
\caption{Minimum example of how an experiment can be specified and run. Here, \texttt{AgentA} and \texttt{AgentB} are agent implementations that the user has written, and \texttt{MyNeuralProblem} is defined like in \ref{lst:metrics} but with an additional metric named \texttt{output}. The visualisations \texttt{Scatter2D} and \texttt{Histogram} are provided as part of the framework.}
\label{lst:experiment}
\end{listing}

\section{Implementation}
The top-level package that constitutes this framework is named \texttt{nf} which is short for ``neural framework''.
The framework code is provided in three main modules inside the \texttt{nf} package, which will be explained below.
The main base classes are implemented in the respective \texttt{\_\_init\_\_.py} files so that they can be conveniently loaded using for example the syntax
\mintinline{python}|from nf.agents import Agent|
instead of the more verbose
\mintinline{python}|from nf.agents.agent import Agent|
which would arise when placing the \texttt{Agent} class inside a file named \texttt{agent.py}.
This is common practice for these types of frameworks.

\paragraph{The \texttt{problems} module}
This module contains the abstract \texttt{Problem} class. 
Some problem implementations are provided in submodules, such as the \texttt{StripeProblem} class in the \texttt{stripe\_problem} submodule.

\paragraph{The \texttt{agents} module}
This module contains the abstract \texttt{Agent} class which requires the subclasses to override a \texttt{fit()} method of the same form as in the \texttt{keras} library.
In fact, the agents must support the same lifecycle methods and callback hooks as \texttt{keras} models.
To take care of some of this for the user \texttt{Agent} class is provided alongside two abstract subclasses: \texttt{DerivativeBasedAgent} and \texttt{DerivativeFreeAgent} that provide more support in implementing each of these techniques. 
A total of four agent implementations have been provided (two of each category). 
\begin{figure}
    \centering
    \centerline{
        \begin{tikzpicture}
            \begin{abstractclass}[text width=9cm]{Agent}{-3.5,0}
                \attribute{problem: Problem}
                \operation[0]{compile()}
                \operation[0]{fit(X: Tensor, y: Tensor, epochs: int, callbacks: list)}
                \operation{train(epochs: int, metrics: list): dict}
            \end{abstractclass}
            \begin{abstractclass}[text width=5cm]{DerivativeBasedAgent}{-5.5,-5}
                \inherit{Agent}
                \operation{fit(\dots)}
            \end{abstractclass}
            \begin{abstractclass}[text width=8cm]{DerivativeFreeAgent}{3.5,-5}
                \inherit{Agent}
                % \attribute{sampler : SamplingTechnique}
                \operation{fit(\dots)}
                \operation{compile()}
                \operation{predict\_for\_weights(weights: Tensor, X: Tensor): Tensor}
                \operation{predict\_for\_multiple\_weights(weights: Tensor, X: Tensor): Tensor}
                \operation[0]{choose\_best\_weight\_update(weight\_samples: Tensor, weight\_history: Tensor, output\_history: Tensor, X: Tensor, y: Tensor): Tensor}
            \end{abstractclass}
            \begin{class}[text width=2cm]{MSE}{-7,-11}
                \inherit{DerivativeBasedAgent}
                \operation{compile()}
            \end{class}
            \begin{class}[text width=2cm]{LWGLD}{-4.5,-11}
                \inherit{DerivativeBasedAgent}
                \operation{compile()}
            \end{class}
            \begin{class}[text width=5cm]{GreedyProbing}{-.25,-11}
                \inherit{DerivativeFreeAgent}
                \operation{choose\_best\_weight\_update(\dots)}
            \end{class}
            \begin{class}[text width=5cm]{SimulatedAnnealing}{5.25,-11}
                \inherit{DerivativeFreeAgent}
                \operation{choose\_best\_weight\_update(\dots)}
            \end{class}
            \begin{abstractclass}[text width=5cm]{SamplingTechnique}{5,0}
                \operation[0]{initialize(num\_weights: int)}
                \operation[0]{\_\_call\_\_(weights: Tensor): \dots}
            \end{abstractclass}
            \unidirectionalAssociation{SamplingTechnique}{sampler}{}{DerivativeFreeAgent}
        \end{tikzpicture}
    }
    \caption{UML class diagram of the \texttt{agents} module, showing only the (non-underscored) public methods. Note that ``LossWithGoalLineDeviation'' is abbreviated as LWGLD and the subclasses of \texttt{SamplingTechnique} have been omitted.}
    \label{fig:agents_uml}
\end{figure}
\ref{fig:agents_uml} provides a UML diagram of this module, showing how the implementations are related via inheritance.

The \texttt{agents} module also contains a submodule, \texttt{sampling}, which defines the abstract \texttt{SamplingTechnique} class (this is shown in the diagram as well).
It provides a common way that derivative-free agents can perform sampling in weight space. 
A total of three sampling techniques have been implemented:
the \texttt{ExhaustiveSamplingTechnique} and \texttt{RandomSamplingTechnique} have been explained in \ref{sec:greedy_probing}. 
The third technique, \texttt{RandomSamplingGenerator} is a variant of the random sampling technique that does not require the number of samples to generate up front; instead, random samples are generated on demand using a so-called \texttt{generator} function in Python.
This leads to efficiency benefits in the Simulated Annealing algorithm where it is not known a priori how many weight states need to be sampled before one is accepted.

\paragraph{The \texttt{experiment} module}
This module contains not only the \texttt{Experiment} class, but also two major aspects that are linked closely to the experiment itself: visualisations and metrics.
Let us briefly look at their implementation before examining the \texttt{Experiment} class itself.

The visualisations are achieved using the \texttt{bokeh} library for Python which provides a web-based user interface.
Originally, the classical plotting library \texttt{matplotlib} was used, but issues with regard to facilitating user interaction on plots that were updating in real time ultimately lead to the adoption of \texttt{bokeh}.
The \texttt{visualisations} submodule contains an abstract \texttt{Visualisation} class which provides a common interface for creating viualisations, so that the user may implement custom ones using \texttt{bokeh}.
Two useful types of visualisations are provided with the framework: a two-dimensional scatter plot (\texttt{Scatter2D}) that can plot arbitrary metrics on each axis, and a histogram (\texttt{Histogram}) that will plot data over time (epochs).
Each agent's data is plotted in a unique colour, and that colour is consistent across all visualisations. 
Furthermore, each visualisation supports toggling the visibility of each agent's data\footnote{On the front end, the user has a button to hide each of the agents separately. This is demonstrated in \ref{app:framework_user_manual}.}.

The \texttt{Metric} class is contained in the \texttt{metrics} submodule which is how metrics are internally represented in the framework. 
One interesting feature is the specification of metrics: \ref{lst:experiment} actually showed three different ways that the user may specify metrics for a visualisation.
The default syntax is of the form
\texttt{\textit{<name>}:\textit{<dim $1$>}:\textit{<dim $2$>}:\textit{\dots}:\textit{<dim $n$>}}
where \texttt{\textit{<name>}} is the metric's name and \texttt{\textit{<dim $i$>}} is the index into the $i$th dimension of the rank-$n$ tensor representing the metric data.
In the case of the \texttt{Scatter2D} visualisation which provides a two-dimensional scatter plot, one can also simply omit the last dimension index, in which case the metrics are unfolded among that dimension such that for example
\mintinline{python}|Scatter2D("weights")| is equivalent to \mintinline{python}|Scatter2D("weights:0", "weights:1")|.

Most importantly, this module contains the eponymous class \texttt{Experiment} which is responsible for coordinating all the aforementioned components.
When running the experiment, this class uses a form of round-robin scheduling whereby each agent is run for a specific number epochs at a time parameterised by \texttt{epoch\_batch\_size} before moving on to the next agent.
Once each agent had their turn, the collected metrics are aggregated and sent to update the front end in real time.
This round-robin approach was chosen in favour of letting the agents train in parallel because
\begin{enumerate*}[label=(\roman*)]
    \item running multiple training algorithms in parallel is very computationally expensive, especially for problems with large datasets; and
    \item it would require a mechanism to synchronise the iterations between agents in order to ensure a fair comparison based on epochs, which will likely cause some agents repeatedly have to wait for the others.
\end{enumerate*}

There are two specific considerations that were made with regards to the collection and aggregation of metrics. 
Firstly, a mechanism was developed that determines the minimal set of metrics that are necessary for the visualisations, and computes only these (instead of all) in order to speed up training.
Furthermore, the experiment itself collects some default metrics that can be visualised by the user.
These are
\begin{enumerate*}[label=(\roman*)]
    \item \texttt{epoch} which collects the epoch number at each epoch; and
    \item \texttt{run\_time}\footnote{This metric is, in fact, used as the $x$-axis of the \texttt{Histogram} visualisation.} which collects the average run\_time per epoch\footnote{Since training is carried out in batches of \texttt{epoch\_batch\_size}, the experiment does not know the run time on a per-epoch basis, so it will assign each epoch in the batch the run time of the entire batch divided by \texttt{epoch\_batch\_size}.}.
\end{enumerate*}
Both contain a scalar value per epoch.

\paragraph{Demonstrations}
To supplement the framework, three demonstration scripts were written.
They are located in the parent directory of the framework and showcase how the framework can be used to set up and run experiments (see \ref{app:framework_user_manual}).
Furthermore, screenshots of the demonstrations in action are provided in the \texttt{framework/screenshots} folder of the code submission.

\section{Documentation}
The quality of a framework is quite obviously predicated in part on the quality of its documentation.
All classes and methods in the framework are documented using not only \textit{type annotations} in line with the PEP\footnote{PEP, short for Python Enhancement Proposal, is a collection of design documents containing best practices for developing Python code.} 484 standard, but also \textit{docstrings} according to the PEP 257 standard.
This allowed the automatic generation of comprehensive documentation of the \texttt{nf} package using the \texttt{pdoc} tool. 
The documentation is available in HTML format at \texttt{framework/docs/nf/index.html} and PDF format at \texttt{framework/docs.pdf}.

Furthermore, a user manual is provided in \ref{app:framework_user_manual} of this report and the demonstration scripts mentioned in the previous section can be used as starter code.

